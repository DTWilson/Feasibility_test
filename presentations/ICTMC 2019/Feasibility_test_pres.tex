\PassOptionsToPackage{unicode=true}{hyperref} % options for packages loaded elsewhere
\PassOptionsToPackage{hyphens}{url}
%
\documentclass[
  ignorenonframetext,
]{beamer}
\usepackage{pgfpages}
\setbeamertemplate{caption}[numbered]
\setbeamertemplate{caption label separator}{: }
\setbeamercolor{caption name}{fg=normal text.fg}
\beamertemplatenavigationsymbolsempty
% Prevent slide breaks in the middle of a paragraph:
\widowpenalties 1 10000
\raggedbottom
\setbeamertemplate{part page}{
  \centering
  \begin{beamercolorbox}[sep=16pt,center]{part title}
    \usebeamerfont{part title}\insertpart\par
  \end{beamercolorbox}
}
\setbeamertemplate{section page}{
  \centering
  \begin{beamercolorbox}[sep=12pt,center]{part title}
    \usebeamerfont{section title}\insertsection\par
  \end{beamercolorbox}
}
\setbeamertemplate{subsection page}{
  \centering
  \begin{beamercolorbox}[sep=8pt,center]{part title}
    \usebeamerfont{subsection title}\insertsubsection\par
  \end{beamercolorbox}
}
\AtBeginPart{
  \frame{\partpage}
}
\AtBeginSection{
  \ifbibliography
  \else
    \frame{\sectionpage}
  \fi
}
\AtBeginSubsection{
  \frame{\subsectionpage}
}
\usepackage{lmodern}
\usepackage{amssymb,amsmath}
\usepackage{ifxetex,ifluatex}
\ifnum 0\ifxetex 1\fi\ifluatex 1\fi=0 % if pdftex
  \usepackage[T1]{fontenc}
  \usepackage[utf8]{inputenc}
  \usepackage{textcomp} % provides euro and other symbols
\else % if luatex or xelatex
  \usepackage{unicode-math}
  \defaultfontfeatures{Scale=MatchLowercase}
  \defaultfontfeatures[\rmfamily]{Ligatures=TeX,Scale=1}
\fi
% use upquote if available, for straight quotes in verbatim environments
\IfFileExists{upquote.sty}{\usepackage{upquote}}{}
\IfFileExists{microtype.sty}{% use microtype if available
  \usepackage[]{microtype}
  \UseMicrotypeSet[protrusion]{basicmath} % disable protrusion for tt fonts
}{}
\makeatletter
\@ifundefined{KOMAClassName}{% if non-KOMA class
  \IfFileExists{parskip.sty}{%
    \usepackage{parskip}
  }{% else
    \setlength{\parindent}{0pt}
    \setlength{\parskip}{6pt plus 2pt minus 1pt}}
}{% if KOMA class
  \KOMAoptions{parskip=half}}
\makeatother
\usepackage{xcolor}
\IfFileExists{xurl.sty}{\usepackage{xurl}}{} % add URL line breaks if available
\IfFileExists{bookmark.sty}{\usepackage{bookmark}}{\usepackage{hyperref}}
\hypersetup{
  pdftitle={A hypothesis test of feasibility for external pilot trials assessing recruitment, follow-up and adherence rates},
  pdfauthor={Duncan T. Wilson, Rebecca E.A. Walwyn, Julia Borwn, Amanda J. Farrin},
  pdfborder={0 0 0},
  breaklinks=true}
\urlstyle{same}  % don't use monospace font for urls
\newif\ifbibliography
\setlength{\emergencystretch}{3em}  % prevent overfull lines
\providecommand{\tightlist}{%
  \setlength{\itemsep}{0pt}\setlength{\parskip}{0pt}}
\setcounter{secnumdepth}{-2}

% set default figure placement to htbp
\makeatletter
\def\fps@figure{htbp}
\makeatother


\title{A hypothesis test of feasibility for external pilot trials assessing
recruitment, follow-up and adherence rates}
\author{Duncan T. Wilson, Rebecca E.A. Walwyn, Julia Borwn, Amanda J. Farrin}
\date{Leeds Institute of Clinical Trials Research}

\begin{document}
\frame{\titlepage}

\hypertarget{set}{%
\section{Set}\label{set}}

\begin{frame}{Motivation}
\protect\hypertarget{motivation}{}

\begin{itemize}
\tightlist
\item
  The success of a larege, definitve COmplex intervention RCT can depend
  on things like recruitment, follow-up and adherence rates.
\item
  When these are unknown, we risk running the trial only to find out
  it's infeasible - a waste of time and money.
\item
  To avoid this, we often run a small pilot trial to estimate these
  rates and guide the stop/go decision - so-called progression criteria,
  a pre-soecified decision rule.
\item
  But, these pilot estimates are variable, and choosing a specific
  decision rule and pilot sample size is difficult. How do we ensure
  they are not to lax, or too strict? How do we decide how precise the
  estimates should be? A complex statistical problem.
\end{itemize}

\end{frame}

\begin{frame}{Content}
\protect\hypertarget{content}{}

We will:

\begin{itemize}
\tightlist
\item
  Introduce a formal method for making stop / go decisions based on
  recruitment, follow-up and adherence rates;
\item
  Use a hypothesis testing framework to define a test statistic;
\item
  Inform the choice of decision rule and sample size using the usual
  operating characteristics of type I and II error rates.
\item
  Apply this method to an example, and contrast it with an alternative
  approach.
\end{itemize}

\end{frame}

\hypertarget{body}{%
\section{Body}\label{body}}

\begin{frame}{Set-up}
\protect\hypertarget{set-up}{}

The deifnitive trial will compare the intervention with control in a
two-arm parralell group set-up, with a normally distributed primary
outcome with known variance.

We are iterested in:

\begin{itemize}
\tightlist
\item
  Recruitment rate, \(\phi_r\);
\item
  Follow-up rate, \(\phi_f\);
\item
  Adherence rate, \(\phi_a\).
\end{itemize}

We denote the parameter vector by \(\phi = (\phi_r, \phi_f, \phi_a)\),
and the pilot estimate by
\(\hat{\phi} = (\hat{\phi}_r, \hat{\phi}_f, \hat{\phi}_a)\).

Our problem is to choose an approriate decision rule, \[
d(\hat{\phi}): [0,1]^3 \rightarrow \{stop, go\}
\] together with a pilot sample size, \(n_p\).

\end{frame}

\begin{frame}{Test}
\protect\hypertarget{test}{}

We base progression decisions on a test statistic: the estimated power
of the main trial, denoted \(f(\hat{\phi})\).

Our decision rule requires a cut-off value \(c\): \[
d[f(\hat{\phi})] = 
\begin{cases}
stop ~\text{ if } f(\hat{\phi}) \leq c, \\
go ~~~~\text{ if } f(\hat{\phi}) > c.
\end{cases}
\]

For example, if we set \(c = 0.7\) then \[
\hat{\phi}_1 = (\hat{\phi}_r = 0.6, \hat{\phi}_f = 0.8, \hat{\phi}_a = 0.95) \\
f(\hat{\phi}_1) = 0.6 < c \Rightarrow stop,
\] while \[
\hat{\phi}_1 = (\hat{\phi}_r = 0.6=8, \hat{\phi}_f = 0.9, \hat{\phi}_a = 0.9) \\
f(\hat{\phi}_1) = 0.8 > c \Rightarrow go.
\]

\end{frame}

\begin{frame}{Operating characteristics}
\protect\hypertarget{operating-characteristics}{}

Due to the sampling variation in \(\hat{\phi}\) we might make mistakes.
For example, \[
f(\phi) = 0.9 - \text{ feasible, want to proceed;} (\phi \in \Phi_1) \\
f(\hat{\phi}) = 0.6 < c \Rightarrow stop, ~ \text{ a type II error.}
\] Similarly, \[
f(\phi) = 0.5 - \text{ infeasible, want to stop;} (\phi \in \Phi_0) \\
f(\hat{\phi}) = 0.75 > c \Rightarrow go, ~ \text{ a type I error.}
\]

Given null and alternative hypotheses, we can define error rates in the
usual way: \[
\alpha = \max_{\phi \in \Phi_0} Pr[\hat{\phi} > c ~|~ \phi] \\
\beta = \max_{\phi \in \Phi_1} Pr[\hat{\phi} \leq c ~|~ \phi]
\]

\end{frame}

\begin{frame}{Hypotheses}
\protect\hypertarget{hypotheses}{}

We define out null hypothesis as all those points \(\phi\) which would
lead to a power no more than some value \(p_0\).

Similarly, the alternative hypothesis contains all \(\phi\) which would
give a power of at least \(p_1\).

\[ 
\Phi_0 = \{\phi ~|~ f(\phi) \leq p_0 \} \\
\Phi_1 = \{\phi ~|~ f(\phi) \leq p_1 \}
\]

For example, suppose \(p_0 = 0.6\) and \(p_1 = 0.8\). We can plot the
hypotheses:

\end{frame}

\begin{frame}{Comparator}
\protect\hypertarget{comparator}{}

Having defined hypotheses, we can calculate error rates for any decision
rule.

In particular, consider typical independant progression criteria of the
form \[
d(\hat{\phi}) = 
\begin{cases}
stop ~\text{ if } \hat{\phi}_r \leq c_r, ~\text{ or }~ \hat{\phi}_f \leq c_f, ~\text{ or }~ \hat{\phi}_a \leq c_a \\
go ~~~~\text{ if } \hat{\phi}_r > c_r ~\text{ and }~ \hat{\phi}_f > c_f ~\text{ and }~ \hat{\phi}_a > c_a
\end{cases}
\]

In this case, we now want to choose \(n_p\) and the three progression
criteria thresholds, \(c_r, c_f\) and \(c_a\), based on the error rates
they lead to.

\end{frame}

\begin{frame}{Application}
\protect\hypertarget{application}{}

Conisder a definitive trial with a planned sample size of \(n = 514\)
per arm, which will give 90\% power to detect a (small) standardies
effect size of 0.3, assuming a two-sided type I error rate of 0.05 and
allowing for a 10\% rate of loss to follow-up.

We define our hypotheses by saying that a true power less than
\(p_0 = 0.6\) would be infeasible, and should be avoided; while a true
power of \(p_1 = 0.8\) would be feasible and should not be missed.

\end{frame}

\begin{frame}{Results}
\protect\hypertarget{results}{}

\end{frame}

\begin{frame}{Conclusions}
\protect\hypertarget{conclusions}{}

We have described one formal approach to making stop go decisions after
a pilot, based on estimates of recruitment, follow-up and adherence.

We have shown that, for the way we have defined hypotheses, this test
gives OCs much better than more conventional PCs.

We have assumed known variance, but this can be relaxed; and if only a
sibset of thse parameters are of interest, we can easily reduce the
method down.

\end{frame}

\begin{frame}

Thank you

\begin{itemize}
\tightlist
\item
  Pre-print available on arXiv
\item
  Paper, slides and all code are at
  \url{https://github.com/DTWilson/Feasibility_test}
\item
  Contact: @DTWilson,
  \href{mailto:d.t.wilson@leeds.ac.uk}{\nolinkurl{d.t.wilson@leeds.ac.uk}}
\end{itemize}

\end{frame}

\begin{frame}{External pilot trials}
\protect\hypertarget{external-pilot-trials}{}

\(\{\text{Pilot trials}\} \subset \{\text{Feasibility studies}\}\)
(Eldridge et al., 2016)

A small version of the planned main study.

Asking the questions: \emph{should} we do the main trial, and if so,
\emph{how}?

Common quantitative objectives: estimating recruitment rates, follow-up
rates, and adherence rates (Avery et al., 2017).

\end{frame}

\begin{frame}{Example - REACH}
\protect\hypertarget{example---reach}{}

REACH (Research Exploring Physical Activity in Care Homes): a pilot for
a complex intervention designed to increase the physical activity of
care home residents.

Feasibility outcomes:

\begin{itemize}
\tightlist
\item
  \textbf{Recruitment} (measured in terms of the average number of
  residents in each care home who participate in the trial);
\item
  \textbf{Adherence} (a binary indicator at the care home level
  indicating if the intervention was fully implemented);
\item
  \textbf{Follow-up} (a binary indicator for each resident of successful
  follow-up at the planned primary outcome time of 12 months);
\item
  \textbf{Efficacy} (a continuous measure of physical activity at the
  resident level).
\end{itemize}

\end{frame}

\begin{frame}{Example - REACH}
\protect\hypertarget{example---reach-1}{}

Cluster randomised, with 6 care homes per arm.

Progressing to the definitive trial if the \emph{progression criteria}
are satisfied:

\end{frame}

\end{document}
